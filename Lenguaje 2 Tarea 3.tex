\documentclass[a4,10pt]{article}

\usepackage[margin=1in]{geometry}
\usepackage{fancyhdr}
\usepackage{graphicx}
\usepackage{cancel}
\usepackage[english]{babel}
\usepackage{hyperref}
\usepackage[
backend=biber,
style=ieee,
]{biblatex}

\addbibresource{ref.bib}

\pagestyle{fancy}
\fancyhead[LO,L]{ FINESI}
\fancyhead[CO,C]{Programming Language II}
\fancyhead[RO,R]{\today}
\fancyfoot[LO,L]{Mamani Maraza Rusbel Yampier}
\fancyfoot[CO,C]{}
\fancyfoot[RO,R]{Page. \thepage}
\renewcommand{\headrulewidth}{0.4pt}
\renewcommand{\footrulewidth}{0.4pt}

\title{Creating a Very Basic Programming Language}

\begin{document}
	
	\section{¿Como pienso creár mi lenguaje basico?}
	\begin{itemize}
		\item First, I do some research on compilers: 
		\item Second, I will define the grammar of the language
		\item Third, I create the lexical analyzer
		\item Fourth, I create the parser
		\item Fifth, create the semantic parser
		\item Sixth, create the middle code
		\item Seventh, optimize the language
		\item Eighth, start generating code
	\end{itemize}
	My language will be of the compilation type.
	I'll use the following grammars for my code:
	
	\begin{verbatim}
		<programa> ::= <instrucción> | <instrucción> <programa> | <comentario> <programa>
		<instrucción> ::= <asignación> | <imprimir>
		<asignación> ::= <identificador> "=" <expresión>
		<imprimir> ::= "see" <expresión>
		<expresión> ::= <operando> | <expresión> "+" <operando> | <expresión> "-" <operando> | <expresión> "*" <operando> | <expresión> "/" <operando>
		<operando> ::= <entero> | <no_exacto> | <cadena>
		<entero> ::= "whole"
		<no_exacto> ::= "imp"
		<cadena> ::= "[" <texto> "]"
		<identificador> ::= [a-zA-Z][a-zA-Z0-9]*
		<texto> ::= cualquier secuencia de caracteres
		<comentario> ::= "*| cualquier texto |*"
	\end{verbatim}
	
	\printbibliography
	
\end{document}
